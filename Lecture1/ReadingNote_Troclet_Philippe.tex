\documentclass[11pt]{article}
\usepackage{amssymb}
\usepackage[french]{babel}
\usepackage{fullpage}
\usepackage[utf8]{inputenc}
\usepackage[T1]{fontenc}
\def\titre{}
\def\auteur{}
\def\courriel{}
\makeatletter

\title{LOG8415\\Concepts avanc\'{e}s en infonuagique}

\author{
    Foutse Khomh \\
    S. Amirhossein Abtahizadeh \\
    D\'{e}partement G\'{e}nie Informatique et G\'{e}nie Logiciel \\
    \'{E}cole Polytechnique de Montr\'{e}al, Qu\'{e}bec, Canada \\
    \texttt{foutse.khomh[at]polymtl.ca} \\
    \texttt{a.abtahizadeh[at]polymtl.ca}
}

\date{}

\begin{document}
\maketitle

\section{Identification}

\paragraph{Student's name:} \auteur Troclet Philippe

\paragraph{Date of the reading note:} \today

\paragraph{Author(s):} Brendan Jennings, Rolf Stadler

\paragraph{Title of the article:} \titre Resource Management in Clouds: Survey and Research Challenges

\paragraph{Publication:} Springer Science+Business Media New York 2014

\section{Article}

\paragraph{Keywords:} Cloud Computing, Resource allocation, Resource management, Virtualization, Survey

\paragraph{Concepts and definitions:}
\begin{description}
	\item[Cloud Computing] : partage de ressources entre plusieurs applications, lesquelles sont accédées via internet.
	\item[Public Cloud] : location de ressources à des tiers qui seront facturés en fonction de leur consommation.
	\item[Private Cloud] : infrastructure privée utilisée et maintenue par une unique organisation.
	\item[Hybrid Cloud] : fait d'entendre les capacités de son private cloud en utilisant des ressources mises à disposition par un public cloud
	\item[Gestion des ressources] : allocation de ressources (processeur, stockage, réseau et énergie) à un ensemble d'applications, de façon à remplir les différents 
		objectifs de performances des applications.
	\item[Fournisseur du cloud] gère un ensemble de centres de données tant du point de vue logiciel que matériel. Il s'engage auprès de l'utilisateur à garantir une
		certaine qualité de service.
	\item[Utilisateur du cloud] utilise les abstractions fournies par le fournisseur pour héberger des applications qui seront uilisées par l'utilisateur final. Il 
		se doit de respecter ses engagements auprès de ce dernier.
	\item[Utilisateur final] utilise les applications mises à sa disposition. Il génére donc le traffic et par extension la charge reçue par les applications.
\end{description}
Il convient de préciser que le terme \textit{ressources} fait référence à l'utilisation du processeur, mais aussi à l'utilisation du stockage, de la bande passante ou
encore de l'énergie. Par ailleurs, l'ensemble des mécanismes développés dans la prochaine section ne sont possibles uniquemant grâce aux technologie suivantes: 
la virtualisation (permet le partage de ressources physiques), la migration de machine virtuelle (permet de déplacer une VM d'un serveur à un autre) et l'ajustement 
dynamique de la consommation d'énergie (permet de consommer moins en période d'inactivité).
\paragraph{Summary:}
La base de l'infonuagique est l'allocation de ressources à différents clients de manière à maximiser l'utilisation du matériel tout en respectant les clauses de 
performances sur lesquelles clients et fournisseurs se sont entendus. La gestion des ressources est alors un défi majeur auxquel les fournsseurs sont confrontés. En effet
les centres de données modernes possèdent un très grand nombre de ressources, potentiellement très différentes mais pouvant être liées par des dépendances qu'il faut
respecter. De plus, la charge liée aux applications hébergées est très variable et difficile à prévoir, donc à anticiper. Afin de mieux cerner le problème, les auteurs 
divisent la problèmatique en huit sous-domaines de recherche:
\begin{enumerate}
	\item\label{global scheduling}  administration globale des ressources vitualisées
	\item\label{resource demand}estimation de la demande
	\item\label{resource utilization}estimation de l'utilisation
	\item\label{resource pricing}maximisation des prix et profits
	\item\label{local scheduling}administration locale des ressources du cloud
	\item\label{application scaling}modulation de la capacité des applications
	\item\label{workload management}gestion de la charge
	\item\label{cloud management}création de système de gestion pour le cloud
\end{enumerate}
Il est intéressant de remarquer que ses sous-domaines sont très liés, afin de maximiser les profits, il est primordial de pouvoir estimer correctement la demande et la
charge courante. On ne peut prendre de décision sans mesures fiables pour supporter la mise en place d'algorithme, de raisonnement. De plus, la facturation étant basée 
sur l'utilisation, il est important de pouvoir estimer les ressources consommées.

Le point \ref{global scheduling} réfère à la création ainsi qu'au déplacement des machines virtuelles de l'utilisateur par le fournisseur au cours du temps.
Ainsi qu'à la gestion des requêtes de création (accepter/refuser) en fonction de l'état courant du système. Le point \ref{resource demand} caractérise la 
nécessité d'établir des modèles de prédiction de la demande d'ajuster les ressources allouées en prévision d'une future augmentation de la charge. Le point 
\ref{resource utilization} souligne le besoin de mesures nécessaires pour surveiller l'activité et répartir la charge. Le point \ref{resource pricing} met en évidence 
l'importance de la mise au point d'un modèle de facturation permettant d'augmenter l'utilisation de l'infrastructure en encourageant les clients à consommer durant les
périodes creuses. En particulier, l'utisation de modèles dynamiques semble prometteuse. Le point \ref{local scheduling}, rappelle que plusieurs machines virtuelles 
tournent sur le même serveur, aussi il est important de bien gérer la répartition des ressources (cpu, réseau, ...) afin de remplir les objectifs de performance. Par
ailleurs, le point \ref{application scaling} met en lumière l'importance de pouvoir changer les logiciels et leur confgurations de façon dynamique afin de pouvoir s'adapter
à une augmentation de la demande. De manière plus intuitive, le point \ref{workload management} se rapporte à la gestion de la répartion de requête afin que les objectifs
de performance soient atteints. De même, il peut être nécessaire de refuser certaines requêtes afin de préserver la performance générale. Enfin, le point 
\ref{cloud management} se rapporte au besoin de définir des métriques pour caractériser les objectifs de performances à atteindre.

Précisons qu'au moment de la publication de l'article, la gestion des différentes VM se fait via des méthodes centralisées et l'allocation de ressources se fait de 
manière statique. Toutefois, ces problèmes restent des challenges très complexes, en effet l'allocation de ressources et la migration des machines virtuelles se rapportent
à des problèmes d'optimisation à plusieurs variables. Certaines pouvant être opposées. En effet, il faut prendre en compte l'énergie, le stockage, le cpu, le réseau 
ainsi que des conflits liés à des éléments du processeur comme les caches. Par ailleurs, une fois modélisé, il faut que le problème soit solvable en un temps raisonnable.
A titre d'exemple, le problème d'allocation des ressources est souvent ramené à un problème de sac à dos binaire. Or ce problème est très long à résoudre de façon exacte, 
des heuristiques efficaces sonot donc néssaires.



\paragraph{Research contributions:}

\section{Analysis}

\paragraph{Quality:} ~\newline
\begin{tabular}{p{.25\textwidth}p{.25\textwidth}p{.25\textwidth}p{.25\textwidth}}
\begin{minipage}[t]{.25\textwidth}
General organization:
\begin{description}
\item $\square$ Very good;
\item $\mbox{\ooalign{$\checkmark$\cr\hidewidth$\square$\hidewidth\cr}}$ Good;
\item $\square$ Medium;
\item $\square$ Bad;
\item $\square$ Very bad.
\end{description}
\end{minipage}

&

\begin{minipage}[t]{.25\textwidth}
Language and style:
\begin{description}
\item $\square$ Very good;
\item $\mbox{\ooalign{$\checkmark$\cr\hidewidth$\square$\hidewidth\cr}}$ Good;
\item $\square$ Medium;
\item $\square$ Bad;
\item $\square$ Very bad.
\end{description}
\end{minipage}

&

\begin{minipage}[t]{.25\textwidth}
Technique:
\begin{description}
\item $\square$ Very good;
\item $\mbox{\ooalign{$\checkmark$\cr\hidewidth$\square$\hidewidth\cr}}$ Good;
\item $\square$ Medium;
\item $\square$ Bad;
\item $\square$ Very bad;
\item $\square$ N/A.
\end{description}
\end{minipage}

&

\begin{minipage}[t]{.25\textwidth}
Bibliography:
\begin{description}
\item $\square$ Very good;
\item $\mbox{\ooalign{$\checkmark$\cr\hidewidth$\square$\hidewidth\cr}}$ Good;
\item $\square$ Medium;
\item $\square$ Bad;
\item $\square$ Very bad;
\end{description}
\end{minipage}
\end{tabular}

\paragraph{Forces of the message:}

\paragraph{Weaknesses of the message:}

\paragraph{Future directions:}

\paragraph{Other important articles:}

\end{document}
